\vspace{-0.7em}
\section{Conclusion and future work}
We analyzed  the behavior of two recently proposed congestion control protocols
for data center networks; namely DCQCN (ECN based) and TIMELY (delay based).
Using fluid models and control theoretic analysis we derived stability regions
for DCQCN, which demonstrated a somewhat odd non-monotonic behavior of stability
with respect to the number of contending flows. We verified this behavior via
packet level simulations. We showed that DCQCN converges to a unique
fixed point exponetially. In performing similar analysis for TIMELY, we discovered that as proposed
the TIMELY protocol has infinite fixed points which could lead to unpredictable
behavior and unbounded unfairness. We provide a simple fix to TIMELY to remedy
this problem. The modified protocol is stable, and converges quickly.

However, for both protocols, the operating queue length grows with the number of
contending flows, which can introduce significant latency. Using a PI controller
on the switch to mark packets, we can guarantee bounded delay and fairness for
DCQCN.  However we demonstrate and prove a fundamental uncertainty result for
delay-based protocols: if you use delay as the only feedback signal for
congestion control, then you can either guarantee fairness or a bounded delay,
but not both simultaneously. Based on this reason, and the fact that ECN marking
process on modern shared-buffer switches effectively excludes queuing delay from
feedback loop, we conclude that ECN is a better congestion signal in data center
environment. 

For future work, we are doing a full exploration of PI like controllers for
congestion control of RDMA in the datacenter, including a hardware
implementation. Our analysis also suggests that DCQCN can be simplified
considerably, to remove strange artifacts like the non-monotonic stability
behavior. We also plan to analyze the impact of PFC-induced PAUSES on the two
protocols.


%%% Local Variables:
%%% mode: latex
%%% TeX-master: "main"
%%% End:
