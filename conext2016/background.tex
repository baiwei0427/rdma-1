\vspace{-0.1em}
\section{Background}
The Remote Direct Memory Access (RDMA) technology offers high throughput (40Gbps
or more), low latency (few $\mu$s), and low CPU overhead (1-2\%), by bypassing
the end-host kernels during data transfer. Instead, network interface cards
(NICs) transfer data in and out of pre-registered memory buffers at the two end
hosts.

Modern data center networks deploy RDMA using the RDMA over Converged Ethernet
V2 (RoCEv2)~\cite{rocev2} standard.  RoCEv2 requires a lossless (or, more
accurately, drop-free) L2 layer. Ethernet can be made drop-free using Priority
Flow Control (PFC)~\cite{pfc}. PFC prevents buffer overflow on Ethernet switches
and NICs. The switches and NICs track ingress queues. When the queue exceeds a
certain threshold, a PAUSE message is sent to the upstream entity. The uplink
entity then stops sending on that link till it gets an RESUME message.  PFC is a
blunt mechanism, since it does not operate on a per-flow basis. This leads to
several well-known problems such as head-of-the-line
blocking~\cite{dcqcn,tcp-bolt}. 

PFC's problems can be mitigated using per-flow congestion control. Since PFC
eliminates packet drops due to buffer overflow, either ECN or increase in RTT
are the only two available ``end-to-end'' congestion signals.  DCQCN relies on
ECN, while TIMELY relies on RTT changes.

DCQCN and TIMELY are not RDMA-specific, so our analysis makes no reference to RoCEv2
or RDMA. DCQCN and TIMELY are not designed for wide area traffic, so our
analysis focuses on intra-DC networks. 
