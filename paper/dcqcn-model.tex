\subsection{Model}

A fluid model of DCQCN was described in~\cite{dcqcn}. We reproduce it here in
Tables~\ref{tab:dcqcn_var} and \ref{tab:dcqcn_param} and
Figure~\ref{fig:dcqcn_model}.  The model considers N flows, traversing a single
bottleneck link. The model assumes that DCQCN is triggered before PFC, and hence
ignores the impact of PFC. Equation~\ref{eq:mark} calculates the probability of
a packet getting marked.  Equation~\ref{eq:q} describes the queue behavior.
Equation~\ref{eq:alpha} captures the evolution of alpha.  Equations~\ref{eq:rt}
and \ref{eq:rc} describe the calculation of target and sending rate,
respectively. 
\begin{table}[t]
\center
{
\footnotesize
{
\begin{tabular}{|c|c|} \hline
Variable & Description \\ \hline
$R_c$ & Current Rate \\ \hline
$R_t$ & Target Rate \\ \hline
$\alpha$ & See Equation (\ref{eq:rp_dec}) \\ \hline
$q$ & Queue Size \\ \hline
$t$ & Time \\ \hline
\end{tabular}
}
}
\caption{DCQCN Fluid model variables}
\label{tab:dcqcn_var}
\end{table}
\begin{table}[t]
\center
{
\footnotesize
{
\begin{tabular}{|c|c|} \hline
Parameter & Description \\ \hline
$K_{min}, K_{max}, P_{max}$ & RED marking parameters. \\ \hline
$g$ & See Equation (\ref{eq:rp_dec}) \\ \hline
$N$ & Number of flows at bottleneck\\ \hline
$C$ & Bandwidth of bottleneck link\\ \hline
$F$ & Fast recovery steps (fixed at 5) \\ \hline
$B$ & Byte counter for rate increase\\ \hline
$T$ & Timer for rate increase\\ \hline
$R_{AI}$ & Rate increase step (fixed at 40Mbps)\\ \hline
$\tau *$ & Control loop delay \\ \hline
$\tau '$ & Interval of Equation (\ref{eq:rp_alpha_recover})\\ \hline
\end{tabular}
}
}
\caption{DCQCN Fluid model parameters}
\label{tab:dcqcn_param}
\end{table}
\begin{figure}[h]
\fbox 
{
\begin{minipage}{\columnwidth}
\begin{equation}
\small
p(t) = \left\{ \begin{array}{ll}
{\rm{0,}} & q(t) \le {K_{\min }}\\
\frac{{q(t) - {K_{\min }}}}{{{K_{\max }} - {K_{\min }}}}{p_{\max }}, & {K_{\min }} < q(t) \le {K_{\max }}\\
{\rm{1,}} & q(t) > {K_{\max }}
\end{array} \right.
\label{eq:mark}
\end{equation}
\begin{equation}
\small
\frac{{dq}}{{dt}} = N{R_C}(t) - C
\label{eq:q}
\end{equation}
\begin{equation}
\small
\frac{{d\alpha }}{{dt}} = \frac{g}{{\tau '}}\left( {\left( {1 - {{(1 - p(t - \tau *))}^{\tau '{R_C}(t - \tau *)}}} \right) - \alpha (t)} \right)
\label{eq:alpha}
\end{equation}
\begin{equation}
\small
\begin{split}
\frac{{d{R_T}}}{{dt}} = & - \frac{{{R_T}(t) - {R_C}(t)}}{\tau }\left( {1 - {{(1 - p(t - \tau *))}^{\tau {R_C}(t - \tau *)}}} \right) \\
& + {R_{AI}}{R_C}(t - \tau *)\frac{{{{(1 - p(t - \tau *))}^{FB}}p(t - \tau *)}}{{{{(1 - p(t - \tau *))}^{ - B}} - 1}} \\
& + {R_{AI}}{R_C}(t - \tau *)\frac{{{{(1 - p(t - \tau *))}^{FT{R_C}(t - \tau *)}}p(t - \tau *)}}{{{{(1 - p(t - \tau *))}^{ - T{R_C}(t - \tau *)}} - 1}}
\end{split}
\label{eq:rt}
\end{equation}
\begin{equation}
\small
\begin{split}
\frac{{d{R_C}}}{{dt}} = & - \frac{{{R_C}(t)\alpha (t)}}{{2\tau }}\left( {1 - {{(1 - p(t - \tau *))}^{\tau {R_C}(t - \tau *)}}} \right) \\
 & + \frac{{{R_T}(t) - {R_C}(t)}}{2}\frac{{{R_C}(t - \tau *)p(t - \tau *)}}{{{{(1 - p(t - \tau *))}^{ - B}} - 1}} \\ 
 & + \frac{{{R_T}(t) - {R_C}(t)}}{2}\frac{{{R_C}(t - \tau *)p(t - \tau *)}}{{{{(1 - p(t - \tau *))}^{ - T{R_C}(t - \tau *)}} - 1}}
\end{split}
\label{eq:rc}
\end{equation}
\end{minipage}
}
\caption{DCQCN fluid model}
\label{fig:dcqcn_model}
\end{figure}

\para{Model Validation:}
In~\cite{dcqcn} it was shown that the fluid model shown in
Figure~\ref{fig:dcqcn_model} matches actual hardware implementation. Here we
only show our NS3 packet-level simulations are in agreement with the model. This
is illustrated in Figure~\ref{fig:dcqcn_validation}. We model a simple topology,
in which N senders, connected to a switch, send to a single receiver, also
connected to that switch. DCQCN parameters are set to the values proposed
in~\cite{dcqcn}.  We see the fluid model and the simulator are in good
agreement.


