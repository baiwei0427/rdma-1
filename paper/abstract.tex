\begin{abstract}
Microsoft and Google respectively use DCQCN~\cite{dcqcn} and
TIMELY~\cite{timely} for congestion control in their RDMA deployments. In this
paper, we provide control theoretic analysis of the two protocols. Our primary
goal is to derive the classical convergence criteria; but our analysis also
several surprising behaviors that are not apparent from the descriptions
in~\cite{dcqcn} and ~\cite{timely}. Specifically, we find that $(i)$ DCQCN
becomes unstable as the number of contending flows increase, $(ii)$ TIMELY can
converge to a stable regime where flows can exhibit arbitrary unfairness $(iii)$
TIMELY cannot function well without the use of hardware packet pacers, despite
assertions to the contrary in~\cite{timely}.
\end{abstract}
