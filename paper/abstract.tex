\begin{abstract}
Data center networks, and especially drop-free RoCEV2 networks require efficient
congestion control protocols. DCQCN (ECN-based) and TIMELY (delay-based) are two
recent proposals for this purpose.  In this paper, we analyze DCQCN and TIMELY
using fluid models and simulations, for stability and convergence. Our analysis
uncovers several surprising behaviors of these protocols. For example, we show
that DCQCN exhibits non-monotonic stability behavior, and that TIMELY can
converge to stable regime with arbitrary unfairness. We propose simple fixes
that can alleviate these problems, and describe how to tune protocol parameters.
Finally, using the lessons learnt from the analysis, we address the broader
question: are there fundamental reasons to prefer either ECN or delay for
end-to-end congestion control in data center networks? We argue that ECN is 
a better congestion signal, due to the way modern switches mark packets, and
also due to a fundamental limitation of end-to-end delay-based protocols.
\end{abstract}
