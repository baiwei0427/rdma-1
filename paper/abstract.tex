\begin{abstract}
Data center networks, and especially drop-free RoCEV2 networks require efficient
congestion control protocols.  Recently, two protocols, namely, DCQCN
(ECN-based) and TIMELY (delay-based) were proposed for this purpose.  In this
paper, we analyze DCQCN and TIMELY using fluid models and simulations, for
stability and convergence. Our analysis uncovers several surprising behaviors of
these protocols. For example, we show that DCQCN exhibits non-monotonic
stability behavior, and that TIMELY can converge to stable regime with arbitrary
unfairness. We propose simple fixes that can alleviate these problems. We find
that in general, both protocols perform generally well, especially with proper
tuning.

Finally, using the lessons learnt from the analysis, we address the broader
question: are there fundamental reasons to prefer either ECN or delay as the
congestion signal in data center networks? We argue that ECN may be a better
congestion signal in data center environment, due to the way modern switches
mark packets, and also due to a fundamental limitation of delay-based protocols.
\end{abstract}
