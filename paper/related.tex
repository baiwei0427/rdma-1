\section{Related Work}
\label{sec:related}

There is a vast amount of literature on congestion control, and control
theoretic analysis of congestion control protocols of all kinds (drop-based,
delay-based, and ECN-based). For a short overview see~\cite{srikantbook}. Below,
we discuss only a few representative papers.

In~\cite{dctcp-analysis} and ~\cite{qcn-analysis}, Alizadeh et.al. analyze the
DCTCP~\cite{dctcp} and QCN~\cite{qcn} protocols that DCQCN is derived from. These papers served as useful guideposts during for our work.

Fluid model analysis of TCP under the RED AQM controller, and subsequent
development of the PI controller was reported
in~\cite{misra2000fluid,hollot2001designing}. Our exploration of the PI
controller for DCQCN and TIMELY is guided by results
in~\cite{hollot2001designing}.

The discussion in Section~\ref{sec:discuss} assumes that congestion control is
done on an end-to-end basis, and switches don't do much beyond marking packets.
A number of congestion control protocols where the bottleneck switch or a
central controller plays a more active role have been proposed. For example,
RCP~\cite{dukkipati2006rcp} and XCP~\cite{katabi2002congestion}~require the switches to send more detailed
feedback, while proposals like~\cite{vattikonda2012practical,wilson2011better}
use an omniscient central controller for fine grain scheduling and pFabric~\cite{pfabric}~is a timeout based congestion control that requires switches to sort packets. 

%XCP} and pFabri that


%%% Local Variables:
%%% mode: latex
%%% TeX-master: "main"
%%% End:
