\section{Conclusion and future work}
In this paper we have analyzed two recently proposed congestion control
protocols for RDMA over converged ethernet, DCQCN and TIMELY to both
understand the behavior of the protocols as well as develop parameter
tuning guidelines. Using
fluid models and control theoretic analysis we derived stability regions
for DCQCN, which demonstrated a somewhat odd non-monotonic behavior of
stability with respect to the number of contending flows. We verified
this behavior via packet level simulations. We also demonstrated that DCQCN
converges to a unique fixed point and that with the proper choice of
parameters the convergence is fast. In performing similar analysis
for TIMELY, we discovered that as proposed the TIMELY protocol has
infinite fixed points which could lead to unpredictable behavior and
unbounded unfairness. We provide a simple fix to TIMELY that makes the
behavior similar to DCQCN, with a unique operating point to which the
protocol converges fast. Both protocols however exhibit a behavior that
could be problematic in the datacenter environment (with extremely low
propagation delays), and that is the
operating queue length in both cases grows with the number of
contending flows and they could introduce significant latency. In the
wide area world, a technique to guarantee a bounded queueing delay has
been demonstrated via an AQM mechanism using the PI controller. One of
the primary differences between DCQCN and TIMELY is that while the
former uses ECN marking at the switch, the latter uses delay
measurements at the end hosts to perform congestion control. To that
end, we implement the PI controller on the switch for DCQCN and at the
end host for TIMELY. In doing so, we can guarantee bounded delay and a
fairness for
DCQCN, however we demonstrate and prove a fundamental uncertainty
result for TIMELY (and TIMELY-like protocols): if you use delay as the only
feedback signal for congestion control, then you can either guarantee
fairness or a bounded delay, but not both simultaneously.

For future work, we are doing a full exploration of PI like
controllers for congestion control of RDMA in the datacenter,
including a hardware implementation on the switches. Our control
theoretic analysis also suggests that DCQCN can be simplified
considerably, to remove strange artifacts like the non-monotonic
stability behavior and that is the subject of our investigation as well.


%%% Local Variables:
%%% mode: latex
%%% TeX-master: "main"
%%% End:
