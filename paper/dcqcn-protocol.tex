DCQCN is an end-to-end, rate-based congestion control protocol that relies on
ECN~\cite{ecn}. It combines elements of DCTCP~\cite{dctcp} and QCN~\cite{qcn}.
DCQCN algorithm specifies behavior of three entities: the sender (called the
reaction point (RP)), the switch, (called the congestion point (CP)), and the
receiver (called the notification point (NP)). We now briefly describe the
protocol; see~\cite{dcqcn} for more details.

\para{CP behavior:} At every egress queue, the arriving packet is
ECN-marked~\cite{ecn} if the queue exceeds a threshold, using a
RED~\cite{red}-like algorithm.

\para{NP behavior:} The NP receives ECN-marked packets and notifies the RP about
it using Congestion Notification Packets (CNP)~\cite{rocev2} Specifically, if a
marked packet arrives for a flow, and no CNP has been sent for this flow in last
N microseconds, a CNP is generated immediately.

\para{RP behavior:} Upon getting a CNP, the RP reduces its current rate ($R_C$)
and updates the value of the rate reduction factor, $\alpha$, like DCTCP, and
remembers current rate as target rate ($R_T$) for later recovery, as follows:
\begin{equation} 
\small 
\begin{aligned} 
R_T &=& R_C,	\\ 
R_C &=& R_C (1 -\alpha/2),  \\ 
\alpha &=& (1-g)\alpha + g, 
\end{aligned} 
\label{eq:rp_dec}
\end{equation} 
If RP gets no feedback for K time units, $\alpha$ is updated as:
\begin{equation} 
\small 
\begin{aligned} 
\alpha &=& (1-g)\alpha, 
\end{aligned}
\label{eq:rp_alpha_recover} 
\end{equation}

Note that $K$ must be larger than the CNP generation timer. Furthermore, RP
increases its sending rate using a timer and a byte counter, in a manner
identical to QCN~\cite{qcn}. The rate increases has two phases, five stages of
so-called ``fast recovery'', where $R_c$ rapidly approaches $R_t$, and then a
gradual additive increase.

Note that that DCQCN does not have slow start. Senders start at line rate, in
order to optimize the common case of no congestion. DCQCN relies on hardware
rate limiters for per-packet rate limiting. 
