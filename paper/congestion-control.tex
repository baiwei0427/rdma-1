\section{The making of a congestion control algorithm}

In the last few years there has been a steady growth of congestion control algorithms such as DCTCP, QCN, DeTail, DCQCN, TIMELY, RC3, .... Any protocol is targeted for a specific environment, uses a different congestion signal and reacts differently to the signal.

Interestingly, TIMELY and DCQCN while being drastically different, both been proposed for RoCE networks. Instead of designing a new congestion control algorithm, we beg the question of: \textit{what is the difference between the two algorithms?} We answer this question by taking a fundamental approach providing insights on crucial components of any congestion control algorithm and use TIMELY and DCQCN as two examples. 

\subsection{Nuts and Bolts}
Below we summarize the nuts and bolts to any congestion control algorithm including TCP. In section \fixme{X} we show what happens when an algorithm like DCQCN or TIMELY ignores any of these fundamentals. We hope this approach light the path for future congestion control designs.

Timely paper is emphasizing on high precision delay as a good congestion signal,
we should emphasize that RTT, when precisely measured, is still an indicator of
network congestion and that a congestion control mechanism can be designed using
RTT. we should downplay the exact algorithmic contributions of timely but there
are other algorithms that have been studied very well in literature based on RTT
such as FAST, Vegas, and NV.

also mention that from a pragmatic view point, generally in experiments the
rates in gradient timely settle at fair shares and are stable. The reason is
probably because there's always some disturbance/noise in experiments, so the
gradient is never perfectly zero. For any non-zero gradient value the AIMD
nature of timely should push the system towards fairness. I don't know how to
show this with stability analysis, but it is certainly something to consider
given that timely is working so well in practice.


Congestion control provides for a fundamental set of mechanisms to maintain
stability, efficiency, and fairness at the same time. Congestion signal: ECN,
RTT.

Dependency on parameters: It is always possible to tune congestion control
parameters based on some knowledge of the environment and the application
scenario. However, ... the fundamental challenge is whether it is possible to
define one congestion control mechanism that operates reasonably well in a whole
range of possible scenarios without need to tweak parameters on the fly...


On stability: control theory is a mathematical tool for describing dynamic
systems. It lends itself to modeling congestion control -- TCP is a perfect
example of a typical closed loop system that can be described in control
theoretic terms. (say something about fluid model here). In control theory,
there is a mathematically defined notion of system stability.  In a stable
system, for any bounded input over any amount of time, the output will also be
bounded.  For congestion control, what is actually meant by global  stability is
typically asymptotic stability: a mechanism should converge to a certain state
irrespective of the initial state of the network. Local stability means that if
the system is perturbed from its stable state it will quickly return toward the
locally stable state.


On fairness: 


On performance:

Congestion signal:

Feedback:

