\section{Discussion}

We next prove another result that gives a fundamental tradeoff between
fairness and guaranteed delay for protocols that
rely on delay measurements at the end points to implement congestion
control.
\begin{thm}[Fairness/Delay tradeoff]
For congestion control mechanisms that relay purely on end to end
delay measurements, you can either have fairness or a guaranteed delay
bound, but not both simultaneously.
\end{thm}
\begin{proof}
Suppose you have N flows sharing a link of capacity $C$. Then every flow
should distributedly arrive at a rate $C/N$ and the flows need to know
this $N$. If we use end to end delay as the only signal, then this delay
has to carry information about $N$ and hence the converged delay
depends on $N$ and cannot be guaranteed independently. Conversely, if we implement a guaranteed delay congestion
control scheme at the end points, they
will converge to any rate $R_i$ such that $\sum_{i=0}^{N}R_i = C$
making the derivative of measured delay 0 and the actual delay equal
to the guarantee. Since the guarantee is set independent of $N$, the
actual delay contains no information on N and 
thus such a scheme cannot ensure fairness.
\end{proof}
