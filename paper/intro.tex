\section{Introduction}

Large cloud service providers like Microsoft and Google are turning to Remote
DMA (RDMA) technology to support their most demanding
applications~\cite{dcqcn,timely,tcp-bolt,erasure-storage,farm,pilaf}.  The
reason is that RDMA offers significantly higher bandwidth and lower latency than
the traditional TCP/IP stacks, while minimizing CPU
overhead~\cite{dcqcn,farm,timely}. 

In modern data centers, RDMA is deployed using the ROCEv2 standard~\cite{rocev2}.
For efficient operation, ROCEv2 needs to eliminate packet drops due to buffer
overflow. This is achieved by enabling Priority Flow Control (PFC)~\cite{pfc},
and by using per-flow congestion control~\cite{dcqcn,timely}. 

Since PFC eliminates packet drops, any end-to-end congestion control protocol
for RoCEv2 networks must use either ECN markings, or delay as the congestion
signal. Last year, two protocols were proposed for this purpose, namely
DCQCN~\cite{dcqcn} and TIMELY~\cite{timely}. DCQCN uses ECN marking as a
congestion signal, while TIMELY measures changes to end-to-end delay.

% what we do.

In this paper, we analyze DCQCN and TIMELY for stability, convergence and
fairness. More importantly, using the insights drawn from the analysis, we
answer the broader question: are there fundamental reasons to prefer either ECN
or delay as the congestion signal in such environments?

% why we do it.

One of our motivations for analyzing these protocols is their potential for
widespread deployment, beyond reported deployments at Microsoft~\cite{dcqcn} and
Google~\cite{timely}. DCQCN is implemented entirely in Mellanox NICs, and
drivers both Windows and Linux operating system are available. Thus, anyone can
deploy DCQCN in their data centers. While Google's implementation of TIMELY is
not publically available, the core timing functionality it relies on is
supported by most Mellanox NICs and their drivers. 

Apart from this practical consideration, we are also driven by
scientific curiosity. Numerous ECN and delay based congestion control protocols
have been proposed and analyzed before~\cite{ecn, tcp-vegas,
Hollot:PIController,misra2000fluid,hollot2001designing,misra:TAC2002,gorinsky2004feedback}.
However, barring a few recent exceptions~\cite{dctcp-analysis, qcn-analysis}
most prior work has been in the context of wide area networks.  Unfortunately,
neither ECN, nor delay based congestion control has seen widespread
deployment in the wide area Internet, for a variety of reasons. 

However, since RoCEv2 data center networks are drop-free, ECN or
delay-based congestion control is not just useful --- it is necessary. These
networks are a very different from the wide area Internet: the links have far
higher bandwidths, propagation delays are much lower, and the traffic is highly
bursty.  

Thus, the time is ripe for analytical exploration of ECN and delay-based
congestion control protocols in this context; specifically to answer the
question we posed earlier.

% Methodology.

To this end, we analyze DCQCN and TIMELY using detailed fluid models and NS3
simulations. Our methodology is similar to the methodology used ny Alizadeh et.
al. for analyzing DCTCP~\cite{dctcp-analysis} and QCN~\cite{qcn-analysis}. Fluid
models are useful for analyzing properties such as stability, and for rapid
exploration of parameter space.  Simulations are useful for studying
packet-level dynamics and establishing the fidelity of the fluid model.  We will
publish our simulation code, as well the Matlab code used to solve the fluid
models.

% summary of contributions and findings.

We stress that it is not our goal to do a comparison of DCQCN and
TIMELY.  Such comparison makes little sense, since both protocols offer several
tuning knobs, and given a specific scenario, either protocol can be made to
perform as well as the other. Instead, we focus on understanding the basic
behavior of delay based and marking based protocols, understand the
tradeoffs and draw generic insights from the analysis.
Our key contributions, and findings are summarized as follows.

\para{DCQCN:} $(i)$ We extend the fluid model proposed in~\cite{dcqcn}, to
show that DCQCN has a unique fixed point, where flows converge to their fair
share. $(ii)$ We show that while DCQCN is generally stable around this
fixed point, as long as the feedback latency is low. The relationship between
stability and the number of competing flows is non-monotonic, which is very
different from TCP's behavior~\cite{misra:TAC2002}. DCQCN is stable for both
very small, and very large number of flows, and tends to be unstable in between;
especially if the feedback latency is high.  $(iii)$ Using a discrete model, we
who that for typical parameter values, DCQCN converges to the aforementioned
fixed point within 100ms.

\para{TIMELY:} $(i)$ We develop a fluid model for TIMELY, and validate it using
simulations. $(ii)$ The fluid model reveals that since TIMELY relies on {\em RTT
gradient} as congestion signal, it can have infinite fixed points, resulting in
arbitrary unfairness.  $(iii)$ We propose a simple fix to remedy this problem,
and show that the modified version of TIMELY is stable, and converges to a fixed
point within \fixme{xxx}. 

\para{General results:} $(i)$ We show that for both variants of
the protocols studied, the fixed point of the queue length (and hence
feedback delay) increases with the number of flows $(ii)$ Higher
delays are undesirable both from an application perspective as well as
difficulty in controlling the loop. To fix that, we show that using
integral control, e.g. the PI~\cite{Hollot:PIController}-like
marking scheme, we can make 
the queue length (and hence delay) independent of the number of contending
flows. $(iii)$ In controlling the queue length to a fixed point, we
derive a \emph{fundamental} result for congestion control: for a
distributed protocol that uses only delay as the feedback signal, you
can achive either fairness or a guaranteed delay, but not both simultaneously. For a
protocol that employs ECN marking based network feedback, you can
achieve both. $(iv)$ We discuss the pros and cons of the use of hardware rate limiters
(DCQCN uses then, TIMELY does not).

