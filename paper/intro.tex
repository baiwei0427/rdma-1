\section{Introduction}

Large cloud service providers are turning to Remote DMA (RDMA) technology to
support their most demanding applications~\cite{dcqcn,timely,tcp-bolt}.  RDMA
offers significantly higher bandwidth and lower latency than the traditional
TCP/IP stack, while minimizing CPU overhead~\cite{dcqcn,farm,timely}. 

Today, RDMA is deployed using the ROCEv2 standard~\cite{rocev2}. To ensure
efficient operation, ROCEv2 uses Priority Flow Control (PFC)~\cite{pfc} to
prevent packet drops due to buffer overflow.  Since there are no packet drops,
any end-to-end congestion control protocol for RoCEv2 networks must use either
ECN markings, or delay as the congestion signal.  Last year, two protocols were
proposed for this purpose, namely DCQCN~\cite{dcqcn} and TIMELY~\cite{timely}.
DCQCN uses ECN marking as a congestion signal, while TIMELY measures changes to
end-to-end delay.

% what we do.

In this paper, we analyze DCQCN and TIMELY for stability, convergence and
fairness. Using the insights drawn from the analysis, we wish to answer the
broader question: are there fundamental reasons to prefer either ECN or delay as
the congestion signal in data center networks?

% why we do it.

One of our motivations for analyzing DCQCN and TIMELY is their potential for
widespread deployment, beyond large cloud providers~\cite{dcqcn, timely}.  While
both were developed for RoCEv2 networks, there is nothing RDMA-specific about
them, apart from the fact that they both ignore the impact of packet drops. Like
DCTCP~\cite{dctcp}, they are useful in any data center environment.  DCQCN is
implemented entirely in Mellanox NICs, and drivers both Windows and Linux
operating system are available.  Thus, anyone can deploy DCQCN in their data
centers. While TIMELY is not publically available, the core timing functionality
it relies on is supported by many modern NICs and their drivers. Our analysis
provides an understanding of the pros and cons of the two options, and offers
guidance on parameter tuning.

Apart from this practical motivation, we are driven by plain curiosity. There is
plenty of prior work on analysis of ECN and delay based congestion control
protocols~\cite{ecn,
tcp-vegas,misra2000fluid,hollot2001designing,misra:TAC2002,gorinsky2004feedback}.
However, barring a few recent exceptions~\cite{dctcp-analysis, qcn-analysis}
most prior work has been in the context of wide area networks. DCQCN and TIMELY
are designed for data center networks, which is a very different environment
than the wide area Internet -- the links have far higher bandwidths, propagation
delays are much lower, and the traffic is highly bursty, and more likely to get
synchronized.  We want to understand how ECN and delay-based congestion control
protocols perform in data center networks; specifically to answer the question
we posed earlier.

As an added impetus, we note that neither ECN, nor delay based congestion
control has seen widespread deployment in the wide area Internet, for a variety
of reasons.  However, since RoCEv2 data center networks are drop-free, ECN or
delay-based congestion control is not just useful --- it is necessary. Thus, the
time is ripe to study the performance of DCQCN and TIMELY in detail.

% Methodology.

To this end, we analyze DCQCN and TIMELY using fluid models and NS3~\cite{NS3}
simulations.\footnote{We will publish our NS3 and MATLAB code.} Fluid models are
useful for analyzing properties such as stability, and for rapid exploration of
parameter space.  Simulations are useful for studying packet-level dynamics and
establishing the fidelity of the fluid model. 

% summary of contributions and findings.

Before we proceed, we want to stress two things.  First, we limit our discussion
to ECN and delay as congestion signals, since they are supported by current
hardware. Protocols that require switches to do
more~\cite{katabi2002congestion,rcp,pfabric} or use a central
controller~\cite{deadline,perry2014fastpass} are outside the scope of this
paper.

Second, it is not our goal to do a direct comparison of performance of DCQCN and
TIMELY.  Such comparison makes little sense, since both protocols offer several
tuning knobs, and given a specific scenario, either protocol can be made to
perform as well as the other. Instead, we focus on the core behavior the two
protocols to obtain broader insights.  Our key contributions, and findings are
summarized as follows.

\para{DCQCN:} $(i)$ We extend the fluid model proposed in~\cite{dcqcn}, to show
that DCQCN has a unique fixed point, where flows converge to their fair share.
Using a discrete model, we also derive an upper  bound on convergence time.
$(ii)$ We show that DCQCN is stable around this fixed point, as long as the
feedback latency is low. The relationship between stability and the number of
competing flows is, strangely, non-monotonic, which is very different from TCP's
behavior~\cite{misra:TAC2002}. DCQCN is stable for both very small, and very
large number of flows, and tends to be unstable in between; especially if the
feedback latency is high.

\para{TIMELY:} $(i)$ We develop a fluid model for TIMELY, and validate it using
simulations. The model reveals that since TIMELY relies on {\em RTT
gradient} as congestion signal, it can have infinite fixed points, resulting in
arbitrary unfairness.  $(ii)$ We propose a simple remedy,
and show that the modified version is stable, and converges to a fixed point. We
also derive an upper bound on convergence time.

\para{ECN or Delay:} Based on lessons learnt from analysis of DCQCN and TIMELY,
we conclude that ECN is a better signal for conveying congestion information.
One reason for this is that modern shared buffer switches mark packets with ECN
{\em at egress}, effectively decoupling the queuing delay from feedback delay.
This improves system stability. Second, we present a \emph{fundamental} result:
for a distributed protocol that uses only delay as the feedback signal, you can
achieve either fairness or a guaranteed delay, but not both simultaneously. For
an ECN-based protocol you can achieve both by using a
PI~\cite{hollot2001designing}-like marking scheme. This is not possible with
delay-based congestion control.

%% \para{General results:} $(i)$ We show that for both variants of the protocols
%% studied, the fixed point of the queue length (and hence feedback delay)
%% increases with the number of flows $(ii)$ Higher delays are undesirable both
%% from an application perspective as well as difficulty in controlling the loop.
%% To fix that, we show that using integral control, e.g. the
%% PI~\cite{hollot2001designing}-like marking scheme, we can make the queue length
%% (and hence delay) independent of the number of contending flows. $(iii)$ In
%% controlling the queue length to a fixed point, we derive a \emph{fundamental}
%% result for congestion control: for a distributed protocol that uses only delay
%% as the feedback signal, you can achieve either fairness or a guaranteed delay,
%% but not both simultaneously. For a protocol that employs ECN marking based
%% network feedback, you can achieve both. $(iv)$ We discuss the pros and cons of
%% the use of hardware rate limiters (DCQCN uses them, TIMELY does not).


%%% Local Variables:
%%% mode: latex
%%% TeX-master: "main"
%%% End:
