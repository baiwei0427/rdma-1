\subsection{Protocol}

TIMELY is an end-to-end, rate-based congestion control algorithm that uses
changes in RTT as a congestion signal.  It relies on NIC hardware to ontain
fine-grained RTT measurements. RTT is estimated once per completion
event~\cite{rocev2}, which signals the successful transmission of a chunk
(16-64KB) of packets. Upon receiving a new RTT sample, TIMELY computes new rate
for the flow.   The rate computation engine considers three regimes. First, if
the new RTT sample is below a certain threshold ($T_{low}$), TIMELY increases
send rate additively by $\delta$. If the new sample is above a threshold
($T_{high}$), rate is decreased multiplicatively by $\beta$. If the new sample
is between $T_{low}$ and $T_{high}$, the rate change depends on the RTT
gradient.  The gradient is defined as the change between two successive RTT
samples. The value of gradient is smoothed, (EWMA), and normalized. If the
resulting value is positive (i.e. RTT is increasing), sending rate is reduced
multiplicatively by $\beta \times normalized_gradient$.  Otherwise, it is
increased additively by $\delta$.

Note that TIMELY does not use hardware rate limiters, instead, rate control is
done by spacing chunks of data Flows do not start at line rate. If there are N
active flows at a sender, a new flow starts at rate C/(N+1), where C is the
interface link bandwidth.

\begin{algorithm}[t]
\footnotesize
{
\begin{algorithmic}[1]
%\Procedure{CalcRate}{$newRTT$}
\State $newRTTDiff \gets newRTT - prevRTT$
\State $prevRTT \gets newRTT$
\State $rttDiff \gets (1-\alpha) \cdot rttDiff + \alpha \cdot newRTTDiff$
\State $rttGradient = rttDiff/D_{minRTT}$
\If {$newRTT < T_{low}$}
        \State $rate \gets rate + \delta$
\ElsIf {$newRTT > T_{high}$}
        \State $rate \gets rate \cdot  (1 - \beta \cdot (1 - T_{high}/newRTT))$
\ElsIf {$rttGradiant \le 0$}
        \State $rate \gets rate + \delta$
\Else
        \State $rate \gets rate \cdot (1 - \beta \cdot rttGradient)$
\EndIf 
%\EndProcedure
\end{algorithmic}
}
\caption{TIMELY rate calculation}
\end{algorithm}

